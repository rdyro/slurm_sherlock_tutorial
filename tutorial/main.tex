\documentclass[10pt,mathserif, notes]{beamer}

% Attribution: template taken from EE 364b at Stanford, 2018

% include external libraries --------------------------------------------------
\usepackage{graphicx, amsmath, amssymb, psfrag, fancyvrb, listings, amsfonts}
\usepackage{algorithm, algpseudocode}
\usepackage{animate, colortbl, tcolorbox, tikz, cancel, pgfpages, pifont}
\usepackage{hyperref}
\usepackage[outputdir=./build]{minted}

\usepackage[absolute, overlay]{textpos}
\setlength{\TPHorizModule}{1mm}
\setlength{\TPVertModule}{1mm}
\usepackage{caption}
\captionsetup[figure]{font=footnotesize}
% -----------------------------------------------------------------------------

% define additional math symbols ----------------------------------------------
\newcommand{\ones}{\mathbf 1}
\newcommand{\reals}{{\mbox{\bf R}}}
\newcommand{\integers}{{\mbox{\bf Z}}}
\newcommand{\symm}{{\mbox{\bf S}}}  % symmetric matrices

\newcommand{\nullspace}{{\mathcal N}}
\newcommand{\range}{{\mathcal R}}
\newcommand{\Rank}{\mathop{\bf Rank}}

\newcommand{\Tr}{\mathop{\bf Tr}}
\newcommand{\diag}{\mathop{\bf diag}}
\newcommand{\card}{\mathop{\bf card}}
\newcommand{\rank}{\mathop{\bf rank}}

\newcommand{\conv}{\mathop{\bf conv}}
\newcommand{\prox}{\mathbf{prox}}

\newcommand{\Expect}{\mathop{\bf E{}}}
\newcommand{\Prob}{\mathop{\bf Prob}}
\newcommand{\Co}{{\mathop {\bf Co}}} % convex hull
\newcommand{\dist}{\mathop{\bf dist{}}}
\newcommand{\argmin}{\mathop{\rm argmin}}
\newcommand{\argmax}{\mathop{\rm argmax}}
\newcommand{\epi}{\mathop{\bf epi}} % epigraph
\newcommand{\Vol}{\mathop{\bf vol}}
\newcommand{\dom}{\mathop{\bf dom}} % domain
\newcommand{\intr}{\mathop{\bf int}}
\newcommand{\sign}{\mathop{\bf sign}}

\newcommand{\cf}{{\it cf.}}
\newcommand{\eg}{{\it e.g.}}
\newcommand{\ie}{{\it i.e.}}
\newcommand{\etc}{{\it etc.}}
\newcommand{\cmark}{\ding{51}}
\newcommand{\xmark}{\ding{55}}

% -----------------------------------------------------------------------------

\definecolor{mykeyword}{rgb}{0,0.6,0}
\definecolor{mycomment}{rgb}{0,0.6,0.9}
\definecolor{mynumber}{rgb}{0.5,0.5,0.5}
\definecolor{mystring}{rgb}{1,0,0}

% formatting ------------------------------------------------------------------
\mode<presentation>{\usetheme{default}}
\setbeamertemplate{navigation symbols}{}
%\usecolortheme[rgb={0.55,0.08,0.08}]{structure}
\usecolortheme[rgb={0.0,0.0,0.0}]{structure}
\setbeamertemplate{itemize subitem}{--}
\setbeamertemplate{frametitle} {
  \begin{center}
  {\LARGE\bf \insertframetitle}
  \end{center}
}

\newcommand\footlineon{
  \setbeamertemplate{footline} {
    \begin{beamercolorbox}[ht=2.5ex,dp=1.125ex,leftskip=.8cm,rightskip=.6cm]{structure}
    {\footnotesize \insertsection \text{\quad} \insertsubsection}
    \hfill
    {\insertframenumber/\inserttotalframenumber}
    \end{beamercolorbox}
    \vskip 0.45cm
  }
}
\footlineon


%\iffalse
\AtBeginSection[] 
{ 
  \begin{frame}<beamer> 
    \frametitle{Outline} 
  \large
  \tableofcontents[currentsection,currentsubsection] 
    \end{frame} 
} 
%\fi

%\setbeameroption{show notes on second screen=right}
%\setbeamertemplate{note page}{\pagecolor{gray!30}\insertnote}

\DeclareMathOperator*{\minimize}{minimize}
\DeclareMathOperator*{\st}{subject\;to}
\def\begm{\ensuremath\begin{bmatrix}}
\def\endm{\ensuremath\end{bmatrix}}

\setcounter{MaxMatrixCols}{20}

\newcommand{\backupbegin}{
   \newcounter{framenumberappendix}
   \setcounter{framenumberappendix}{\value{framenumber}}
}
\newcommand{\backupend}{
   \addtocounter{framenumberappendix}{-\value{framenumber}}
   \addtocounter{framenumber}{\value{framenumberappendix}}
}

%\newcommand{\code}[1]{\tcbox[nobeforeafter, box align=base,size=fbox]{%
%\texttt{#1}}%
%}
\newcommand{\code}[1]{\tcbox[nobeforeafter, box align=base,size=fbox]{%
\mintinline{bash}{#1}}%
}
% -----------------------------------------------------------------------------

%% begin presentation ---------------------------------------------------------
\title{\LARGE \bfseries Slurm in 5 min.}

\author{\large Robert Dyro\\Stanford ASL}

\date{\today}


\begin{document}

\frame{
  \thispagestyle{empty}
  \titlepage
}

% ############################################################################ %
\section{Introduction}

% ---------------------------------------------------------------------------- %
\begin{frame}[fragile]
\frametitle{Introduction - Overview}
\begin{itemize}
\item a cluster allows to run independent processes across many computers
\item many computers usually means 2 to 20
\item by all means run your computation on only 1 computer if you want!
\item processes are independent, can communicate over network
\item usage through shell through \code{ssh}
\item a cluster is managed by a cluster software - Slurm
\item computers in the cluster communicate via network (sockets)
\item storage is shared, all computers see the same files
\item sharing storage is managed by the shared storage software 
\item no \code{sudo}, install locally or load available packages
\end{itemize}

\end{frame}

\begin{frame}[fragile]
\frametitle{Introduction - Terminology}
\begin{itemize}
\item \emph{a node} - a computer on the network managed by Slurm [-N]
\item \emph{a task} - a single of the many processes you want to run [-n]
\item \emph{a CPU} - a single core on a computer (a physical core) [-c]
\item \emph{a job} - a set of independent processes to run
\item \emph{to submit} - to ask Slurm to run a job
\item \emph{to schedule} - what Slurm does to execute all your tasks
\item \emph{array job} - an alternative way of scheduling NOT covered here
\end{itemize}
\end{frame}

\begin{frame}[fragile]
\frametitle{Introduction - How it works}
\begin{itemize}
\item you submit a job
\begin{itemize}
\item your job consists of instruction how many and what processes to run
\item you specify strict requirements: nodes nb, memory per node, time limit
\item you specify other requirements and job hints
\end{itemize}
\item your jobs waits in a queue until resources (nodes) are available
\begin{itemize}
\item if it is rejected, it is rejected immediately
\end{itemize}
\item your job is scheduled
\begin{itemize}
\item you can get an email about it
\item you check on the status by examining a combined output file
\end{itemize}
\item your job finishes or is cancelled (by you)
\end{itemize}
\end{frame}
% ---------------------------------------------------------------------------- %

% ############################################################################ %
\section{Using Slurm}

% ---------------------------------------------------------------------------- %
\begin{frame}
\frametitle{Important Commands}

\underline{Slurm user commands} (you're the user, as opposed to an admin)
\begin{itemize}
\item \code{sinfo} - query available computational resources
\item \code{squeue} - check status of jobs (\code{squeue --user $USER})
\item \code{srun} - run a single process, low-level command
\item \code{sbatch} - run a job, a job is a shell script (that calls \code{srun})
\end{itemize}
\vspace*{1cm}

\underline{\code{module} - allows to load packages}, no \code{sudo apt} on a cluster
\begin{itemize}
\item \code{module avail} - available preinstalled packages 
\item \code{module list} - currently loaded packages
\item \code{module load package} - load \emph{package}
\item \code{module unload package} - unload \emph{package}
\end{itemize}

\end{frame}

\begin{frame}[fragile]
\frametitle{Example Job Script}
run by issuing \code{sbatch my_job.sh}

\begin{tcolorbox}[size=fbox]
\begin{minted}{bash}
#!/bin/bash
#SBATCH --output=out_%j.txt
#SBATCH --time=0-10:00
#SBATCH --mail-type=ALL
#SBATCH --ntasks=20
#SBATCH --cpus-per-task=2
module load julia # load needed packages
source environment # like a python virtualenv
echo ""; date; echo ""
srun -n1 -c2 python3 setup.py # 1 node, task and cpu
wait # barrier ##########################################
for i in {0..100}; do # 101 times, more than ntasks is OK
  srun -n1 -c2 python3 exp.py $i & # notice &
done # each srun ends with & to run all in the background
wait # barrier ##########################################
echo ""; date; echo ""
\end{minted}
\end{tcolorbox}

\end{frame}

\begin{frame}[fragile]
\frametitle{Scheduling}

\begin{itemize}
\item when you connect, you're on the login node, don't run stuff there
\item \code{srun --pty bash} - to run a computational node interactively
\item \code{srun -p gpu --gpus 1 --pty bash} - a gpu node interactively
\item \tcbox[nobeforeafter, box align=base,size=fbox]{\texttt{\#SBATCH -p gpu}} in the \code{my_job.sh} file to have GPU nodes
\item \code{tail -n 100 out_43598734.txt} to check on a job
\end{itemize}
\end{frame}
% ---------------------------------------------------------------------------- %


% ############################################################################ %
\section{Stanford Sherlock}
\begin{frame}
\frametitle{Sherlock - Stanford University Cluster}
All info at: \underline{\href{https://www.sherlock.stanford.edu}{www.sherlock.stanford.edu}}

\vspace*{1cm}

\begin{itemize}
\item \code{ssh $USER@login.sherlock.stanford.edu}\\
\item \underline{\href{https://login.sherlock.stanford.edu}{https://login.sherlock.stanford.edu}} - web-based interface
\end{itemize}

\vspace*{1cm}

\underline{Comments}
\begin{itemize}
\item uses Slurm
\item access via a linux shell without \code{sudo} privileges
\item has 100s general computational nodes and $\sim$100 GPU nodes
\item account is free, but requires Sherlock admin's and your PI's approval
\item usage is free for research, requires publication acknowledgement
\item can request all the nodes, but will spend time in queue
\item simple job wait: 1 node - 10 s, 10 nodes - 5 min, 20 nodes - 1 hour
\end{itemize}
\end{frame}

\begin{frame}
\frametitle{Sherlock Storage}

\underline{Storage}
\begin{itemize}
\item three types of storage - \emph{long}, \emph{short} and \emph{very short} term
\item storage is very generous (a lot of GB or TB)
\item \emph{short} and \emph{very short} storage locations are vastly faster
\item \emph{short} storage locations have file expiration (90 days)
\item \emph{very short} storage location is cleared on job end
\end{itemize}

\underline{Available storage locations} - these are bash variables
\begin{itemize}
\item \emph{long} term storage
\begin{itemize}
\item \code{$HOME} - personal long term storage
\item \code{$GROUP_HOME} - shared research group storage (your lab)
\end{itemize}
\item \emph{short} term storage
\begin{itemize}
\item \code{$SCRATCH} - personal short term storage
\item \code{$GROUP_SCRATCH} - shared research group storage (your lab)
\end{itemize}
\item \emph{very short} term storage - very good performance
\begin{itemize}
\item \code{$L_SCRATCH} - local to node
\item \code{$L_SCRATCH_JOB} - local to node \& job
\end{itemize}
\end{itemize}

\end{frame}




\begin{frame}[fragile]{Sherlock Tips}

\underline{Persistent login}

Under \code{~/.ssh/config} put
\begin{tcolorbox}[size=fbox]
\begin{minted}{bash}
Host login.sherlock.stanford.edu
    ControlMaster auto
    ControlPath ~/.ssh/%l%r@%h:%p
\end{minted}
\end{tcolorbox}

\underline{Multiple terminal windows at computational node}
\begin{itemize}
\item query what nodes are yours \code{squeue --user $USER}
\item issue \code{ssh sh02-01n43}
\end{itemize}

\underline{Installing things}
\begin{itemize}
\item Sherlock does not give \code{sudo} access
\item many installations do not \emph{really} require \code{sudo}
\item install and use stuff from \code{~/.local/lib} \code{~/.local/bin}
\end{itemize}

\underline{Copying files back and forth}
\begin{itemize}
\item use \code{rsync}! excellent tutorial from Digital Ocean \href{https://www.digitalocean.com/community/tutorials/how-to-use-rsync-to-sync-local-and-remote-directories}{\underline{here}}
\end{itemize}

\end{frame}


\end{document}
% -----------------------------------------------------------------------------
